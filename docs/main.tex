\input eplain

\newdimen\singlespace\relax\singlespace=\baselineskip
\newdimen\doublespace\relax\doublespace=2\singlespace
\newdimen\triplespace\relax\triplespace=3\singlespace

\def\startsinglespace{\baselineskip=\singlespace}
\def\startdoublespace{\baselineskip=\doublespace}

\parindent=0em
\parskip=\singlespace

\newfam\bbfam
\font\bbtwelve=msbm10 at 12pt
\font\bbnine=msbm7 at 9pt
\font\bbseven=msbm5 at 7pt
\textfont\bbfam=\bbtwelve
\scriptfont\bbfam=\bbnine
\scriptscriptfont\bbfam=\bbseven
\def\bb{\fam=\bbfam}

\def\powerset#1{{\cal P}(#1)}
\def\justbelow#1{\lfloor #1 \rfloor}
\def\downset#1{\downarrow\!#1}

\everyfootnote = {%
	\advance\parindent by 1em\relax
	\startsinglespace}
\interfootnoteskip=\singlespace
\let\footnote=\numberedfootnote

\def\truenull{\null\vskip-\baselineskip}
\def\noskip{\truenull\vskip-\baselineskip}
\def\trueskip#1{\noskip\vskip#1}

\def\tableword{Table}
\newbox\capbox
\newcount\tabnum
\def\table#1#2#3{
	\global\advance\tabnum by 1
	\begingroup
	\setbox\capbox = \vbox{
		\startsinglespace
		\leftskip=0pt plus 1fil
		\rightskip=\leftskip
		\parfillskip=0pt
		\noindent\tableword~\the\tabnum: #3
	}
	\vskip\singlespace
	\vbox{
		\hbox to \hsize{\hss#2\hss}
		\box\capbox
	}
	\trueskip\triplespace
	\endgroup
}

\begingroup
\baselineskip=2\baselineskip
\centerline{\bf A simple approach to computing Assembly Indices}
\centerline{D.G. Moore, S. Marshall, Y. Liu, S.I. Walker, L. Cronin}
\centerline{\today}
\endgroup

\vskip 3\baselineskip

Recall that assembly spaces $(\Gamma, \phi)$ are defined in the
supplementary information as a quiver (multigraph) $\Gamma$ together with
an edge labeling $\phi$ which satisfies a few axioms. Where possible,
we will simply write $\Gamma$. The vertices of the assembly space
$V(\Gamma)$ represent objects, e.g. strings, molecules, etc..., and
edges $E(\Gamma)$ represent a method of constructing that object in a
single joining operation from previously constructed (or {it a priori})
objects. The {\it a priori} objects are referred to as basic objects,
and the set of basic objects is referred to as the {\it basis of the
assembly space}.

The definitions that follow are labeled alphabetically. References
to numerically labeled definitions, lemmas and theorems refer to the
supplementary information.

{\bf Definition a.} Let $a, b \in V(\Gamma)$. We write $a \prec b$ if for
all $c \in V(\Gamma)$ with $a \le c \le b$\footnote{See ({\bf Lemma. 1})
for an explanation of $\le$.}, either $c = a$ or $c = b$.

In the context of an assembly space, $a \prec b$ denotes that $b$ can
be constructed from $a$ in a single step by joining $a$ with something
else in the space.

{\bf Definition b.} Let $a \in V(\Gamma)$, the $\coprod a = \{(u, \phi(u))
\in V(\Gamma) \times V(\Gamma) ~|~ u \prec a \}$. For $S \subseteq
V(\Gamma)$, let $\coprod S = \bigotimes\limits_{a \in S} \coprod a$.

You can think of $\coprod a$ as the set of all pairs of things that can
be joined to make $a$.  Similarly, $\coprod S$ where $S$ is a set of
objects can be thought of as all of the ways that every object in $S$
can be constructed in a single step.

{\bf Definition c.} Let $U \subseteq V(\Gamma)$. $S(U) = \{ a \in U ~|~
a \in \downset{(U \setminus \{a\})} \}$.

As we are building up objects via pairwise joining operations, we need
to be able to identify "shortcuts" when they are possible. For example,
consider making $aabaa$. One possible pairwise joining would be $aab *
aa$. But notice that $aa \le aab$. When we then move on to make $aab$,
we want to consider make sure that if we decided to make $aab$ as $aa *
b$ we don't double count $aa$ -- we already made it in the previous step.

In other words, $S(U)$ determined which elements of $U$ could be used
to make something else in $U$.

\beginsection A recursive expression for (co)assembly indicies

As simply as I can put it$\ldots$

The assembly index of an object $c(x)$ is $1$ greater than the smallest
coassembly of any pair of objects which can be joint to create $x$.

The coassembly index $cc(\sigma, \tau)$ of a set of objects $\sigma$
contingent on a set of ``shortcuts'' $\tau$, is the number of complex
(non-basic) objects in $\sigma$ plus the smallest coassembly of all
objects which can be pairwise joined to make every object in $\sigma$.

{\it I'm not quite satisfied with this way of writing things, but it's
about all I've got at the moment.}
$$
\eqalign{
	c(x) & = cc(\{x\}, \emptyset) \cr
	cc(\sigma, \tau) & = |S \setminus \Gamma_B| +
		\min\limits_{\alpha \in \coprod \sigma}{}{
			cc(J(\alpha) \setminus (\Gamma_B \cup S(J(\alpha)) \cup \tau),\;
			   R(J(\alpha)) \cup \tau)
		}
}
$$
where
$$
J(U) = \bigcup\limits_{(a_1,\ldots,a_n) \in U} \{a_1, \ldots, a_n\}.
$$

You can find an implementation of this for strings at \hfil\break
{\tt https://github.com/dglmoore/Pathways.jl/blob/master/src/assembly.jl}.

\bye
